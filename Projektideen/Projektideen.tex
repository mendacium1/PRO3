\documentclass{article}
\usepackage{geometry}
\geometry{
	a4paper,
	%total={170mm,257mm},
	left=20mm,
	top=30mm,
}

\usepackage{fancyhdr}
\usepackage{tikz}
\usepackage{hyperref}
\usepackage{graphicx}
\usepackage{hyperref}
\usepackage{mdframed}
\usepackage{listings} % Include the listings package
\usepackage{xcolor}   % to define your own colors
\usepackage{subcaption}

\bibliographystyle{unsrt}
\bibliography{references}


\newmdenv[
linecolor=blue, % Color of the border line
backgroundcolor=gray!20, % Background color; "gray!20" means "20% gray"
frametitle=Note, % Title of the frame, delete this line if you don't want a title
skipabove=\baselineskip, % Space above the frame
skipbelow=\baselineskip, % Space below the frame
]{mynote}

% code-snippets:
% Define the color styles you wish to use in the document for the Python syntax highlighting
\lstdefinestyle{mystyle}{
	backgroundcolor=\color{white},   % choose the background color; you must add \usepackage{color} or \usepackage{xcolor}
	commentstyle=\color{green},
	keywordstyle=\color{blue},
	numberstyle=\tiny\color{gray},
	stringstyle=\color{red},
	basicstyle=\ttfamily\footnotesize,
	breakatwhitespace=false,         
	breaklines=true,                 
	captionpos=b,                    
	keepspaces=true,                 
	numbers=left,                    
	numbersep=5pt,                  
	showspaces=false,                
	showstringspaces=false,
	showtabs=false,                  
	tabsize=2
}
\definecolor{LightGray}{gray}{0.9}

\lstset{style=mystyle} % Apply your style globally to the document


\newcommand{\LVA}{Projekt}
\newcommand{\LVAKURZ}{PRO3}
\newcommand{\SEMESTER}{WS 2023/2024}
\newcommand{\UELABEL}{}
\newcommand{\UETITLE}{Projektideen}
\newcommand{\AUTHOR}{Jakob Mayr, Lukas Kaiser, Jonas Pfeiffer}


\title{\vspace{5cm} \LVA\ (\LVAKURZ)\\ \vspace{1cm} \textbf{\UETITLE} \vspace{2.5cm}}
\author{\AUTHOR}
\date{\SEMESTER}

\begin{document}
	
	\pagestyle{fancy}
	
	\maketitle
	
	\tikz [remember picture, overlay] %
	\node [shift={(3.7cm,-4cm)}] at (current page.north west) %
	[anchor=north west] %
	{\includegraphics{fhooe_logo.jpg}};
	
	\tikz [remember picture, overlay] %
	\node [shift={(10cm,-4.8cm)}] at (current page.north west) %
	[anchor=north west] %
	{\includegraphics{si_logo.jpg}};
	
	%\tikz [remember picture, overlay] %
	%\node [shift={(7.2cm,-11.65cm)}] at (current page.north west) %
	%[anchor=north west] %
	%{\includegraphics[scale=0.12]{./img/star_wars_logo_no_background.png}};
	%
	%\pagebreak
	
	\fancyhf{}
	\fancyhead[L]{\LVA\ (\LVAKURZ)}
	\fancyhead[C]{\UELABEL}
	\fancyhead[R]{\SEMESTER}
	\fancyfoot[L]{Seite \thepage\ von \pageref{LastPage}}
	\fancyfoot[R]{\AUTHOR}
	
	\section*{Projektideen}
	\begin{enumerate}
		\item \textbf{Security Audit of a Small Business or University System}: Conduct a thorough security audit of a small business's or your university's IT infrastructure. This could involve vulnerability scanning, assessing the effectiveness of existing security measures, and providing recommendations for improvements.\\
		\item \textbf{Cybersecurity Training Game or Application}: Develop an interactive game or application that teaches basic cybersecurity concepts to users, such as password security, safe browsing habits, and understanding of common cyber threats.\\
		\item \textbf{Incident Response Drill}: Plan and execute a simulated cybersecurity incident for your university's IT infrastructure, followed by a thorough incident response that includes identifying the breach, containing it, and recovering from it.
		
		\pagebreak
		
		\item \textbf{Homelab-Security-Infrastructure}:
		\subsection*{Goal}
		
		To build a scalable and secure home-lab environment that can onboard client systems (both Linux and Windows) and provide critical cybersecurity services like WAF and SIEM.\\
		
		\subsection*{Project Phases}
		\begin{enumerate}
			\item \textbf{Planning and Design:}\\
			Define Objectives: Clearly outline what you want to achieve with each service (WAF, SIEM, etc.).
			Infrastructure Design: Plan the network architecture, including the placement of services, segmentation, and how clients will be onboarded.
			Tool Selection: Choose the appropriate tools and software (e.g., Splunk for SIEM, and a suitable WAF solution).
			\item \textbf{Setting Up the Infrastructure:}\\
			Hardware and Software Setup: Acquire necessary hardware and install required software. This may include virtualization solutions for creating different client environments.
			Network Configuration: Set up the network, ensuring proper segmentation and security measures are in place.
			\item \textbf{Service Implementation:}\\
			WAF Setup: Install and configure the WAF to protect web applications from common threats and attacks.
			SIEM Implementation: Deploy a SIEM solution like Splunk, configure it to collect logs from various sources, and set up dashboards for monitoring.
			\item \textbf{Onboarding Clients:}\\
			Client Preparation: Prepare Linux and Windows systems with necessary configurations for onboarding.
			Integration: Integrate these client systems into the home-lab, ensuring they are properly communicating with the WAF and SIEM services.
			\item \textbf{Testing and Optimization:}\\
			Functionality Testing: Test the services for basic functionality and performance.
			Security Testing: Conduct vulnerability assessments and penetration testing to identify and rectify security gaps.
			Optimization: Fine-tune the services based on the test results for optimal performance and security.
			\item \textbf{Documentation and Reporting:}\\
			System Documentation: Create comprehensive documentation of the setup, configurations, and procedures.
			Findings and Recommendations: Document the findings from testing and provide recommendations for improvements.
			\item \textbf{Presentation:}\\
			Final Presentation: Prepare a presentation detailing the project’s objectives, implementation, challenges, findings, and learning outcomes.
			\item \textbf{Considerations:}\\
			Scalability: Ensure the design allows for easy scaling and addition of more clients or services.
			Security: Maintain a strong focus on security best practices throughout the project.
			Budget and Resources: Be mindful of the budget and resources available for the project.
			\item \textbf{Learning Outcomes:}\\
			Hands-on experience with WAF and SIEM tools.
			Understanding of network configuration and client onboarding.
			Insight into the challenges and considerations of managing cybersecurity services in a network environment.\\
			This project would not only enhance your technical skills but also give you valuable insights into the operational aspects of cybersecurity services. It's a comprehensive way to apply theoretical knowledge in a practical setting.\\
		\end{enumerate}		
	\end{enumerate}
	\label{LastPage}
	
\end{document}